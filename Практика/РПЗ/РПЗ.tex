% !TEX program = xelatex
%----------------------- Преамбула -----------------------
\documentclass[12pt, left=30mm, right=10mm, top=20mm, bottom=20mm]{templateReportBMSTU}

% ---------------------- Документ ----------------------
\begin{document}
	\begin{titlepage}
		\noindent\begin{minipage}{0.05\textwidth}
			\includegraphics[scale=0.3]{1.png}
		\end{minipage}
		\hfill
		\begin{minipage}{0.85\textwidth}\raggedleft
			\begin{center}
				\fontsize{12pt}{0.5\baselineskip}\selectfont 
				\textbf{Министерство науки и высшего образования Российской Федерации
				\\ Федеральное государственное бюджетное образовательное учреждение
				\\ высшего образования
				\\ <<Московский государственный технический университет
				\\ имени Н.Э. Баумана
				\\ (национальный исследовательский университет)>>
				\\ (МГТУ им. Н.Э. Баумана)}
			\end{center}
		\end{minipage}

		\begin{center}
			\fontsize{12pt}{0.1\baselineskip}\selectfont
			\noindent\makebox[\linewidth]{\rule{\textwidth}{4pt}} \makebox[\linewidth]{\rule{\textwidth}{1pt}}
		\end{center}
		\begin{flushleft}
			\fontsize{12pt}{1\baselineskip}\selectfont 
			ФАКУЛЬТЕТ \textbf{ИНФОРМАТИКА И СИСТЕМЫ УПРАВЛЕНИЯ}
			\\ КАФЕДРА \textbf{КОМПЬЮТЕРНЫЕ СИСТЕМЫ И СЕТИ (ИУ6)}
			\\ НАПРАВЛЕНИЕ ПОДГОТОВКИ \textbf{09.XX.XX \ Название направления}
			% бакалавры групп 1-3 и 6 – 09.03.01 Информатика и вычислительная техника;
			% бакалавры групп 4 и 5 – 09.03.03 Прикладная информатика;
			% магистры все – 09.04.01 Информатика и вычислительная техника
		\end{flushleft}

		\begin{center}
			\fontsize{18pt}{\baselineskip}\selectfont
			\textbf{\uline{ОТЧЁТ ПО \ <УЧЕБНОЙ> | <ПРОИЗВОДСТВЕННОЙ> }} % Либо то, либо другое
			\\ \textbf{\uline{ПРАКТИКЕ}}
		\end{center}

		\begin{table}[h!]
			\fontsize{14pt}{0.7\baselineskip}\selectfont
			\begin{signstabular}{p{2cm} p{14.4cm}}
			Студент & \uline{\hfill}
			% % 
			\\ \hfill & \multicolumn{1}{c}{\fontsize{10pt}{\baselineskip}\selectfont \textit{Фамилия имя отчество}}
			\\ \ & \
			\\  Группа &  \uline{\hspace*{0.5cm}}\uline{Группа <Шифр>}\uline{\hspace*{0.5cm}}
			\end{signstabular}
		\end{table}
		
		\begin{table}[h!]
			\fontsize{14pt}{0.7\baselineskip}\selectfont
			\begin{signstabular}{p{3cm} p{13.4cm}}
				Тип практики & \uline{\hfill}
				% Название по учебному плану: Практикум по программированию, 
				% Технологическая практика, Производственная практики, Преддипломная практика и т.п.
				\\ \ & \
				\\ Название предприятия & \uline{\hfill}
				% для учебной практики НУК ИУ МГТУ им. Н.Э. Баумана
			\end{signstabular}
		\end{table}
		\
		\begin{table}[h!]
			\fontsize{14pt}{0.7\baselineskip}\selectfont
			\centering
			\begin{signstabular}[0.7]{p{3.25cm} >  {\centering\arraybackslash}p{4cm} > {\centering\arraybackslash}p{4cm} > {\centering\arraybackslash}p{4cm}}
				Студент & \ & \uline{\hspace*{4cm}} & \uline{\hfill \textbf{<И.О. Фамилия>} \hfill} 
				\\ & \ & \footnotesize (Подпись, дата) & \footnotesize (И.О. Фамилия)
				% дата обязательна, формат даты: хх.хх.20хх
			\end{signstabular}

			\vspace{\baselineskip}

			\begin{signstabular}[0.7]{p{3.25cm} >  {\centering\arraybackslash}p{4cm} > {\centering\arraybackslash}p{4cm} > {\centering\arraybackslash}p{4cm}}
				Руководитель & \ & \uline{\hspace*{4cm}} & \uline{\hfill \textbf{<И.О. Фамилия>} \hfill} 
				\\ & \ & \footnotesize (Подпись, дата) & \footnotesize (И.О. Фамилия)
				% дата обязательна, формат даты: хх.хх.20хх
			\end{signstabular}
		\end{table}

		\begin{flushleft}
		Оценка \ \uline{\hspace*{9cm}}
		\end{flushleft}
		\vfill
		\begin{center}
			\normalsize \textit{<Год> г.}
		\end{center}
	\end{titlepage}

	

	\normalsize
	\setcounter{page}{2}

	\pagebreak

	% Переопределяем название \toc и выводим сам \toc
	\renewcommand{\contentsname}{\normalsize\bfseries\centering СОДЕРЖАНИЕ}
	\small
	\tableofcontents
	\normalsize

	\pagebreak

	\section*{ВВЕДЕНИЕ}
		\addcontentsline{toc}{section}{ВВЕДЕНИЕ}
		ТЕКСТ ВВЕДЕНИЯ
		\\ ТЕКСТ
		\pagebreak

	\section{<Раздел>}

	ПРИМЕР НУМЕРОВАННОГО СПИСКА:
	\begin{enumerate} 
  		\item первый элемент первого уровня содержит список 
		\item  второй элемент первого уровня
	\end{enumerate}
	Пример ссылки на раздел: Подробная информация в разделе \ref{section1}
	\\ ПРИМЕР МАРКИРОВАННОГО СПИСКА:
	\begin{itemize}
		\item Each list item is ...
		\item Lists can be nested ...
			\begin{itemize}
				\item The maximum ...
				\item Switching ...
			\end{itemize}
		\item And so on.
	\end{itemize}
	
	Листинг \thenumberlisting \ - Код программы
	\begin{lstlisting}
		c = x^2
		print('test listing')
	\end{lstlisting}
	\pagebreak

	\section{<Раздел>}  \label{section1}

	\begin{tabular}{ | p{100pt} | r | c | }
		\hline
		ИСЗ & Дата запуска & Масса, кг  \\ \hline
		Спутник-1 & 4 октября 1957 & 83.6 \\
		Спутник-2 & 3 ноября 1957 & 508.3 \\
		Эксплорер-1 & 1 февраля 1958 & 21.5 \\
		\hline
	\end{tabular}

	\begin{figure}[h]
		\centering
		\includegraphics[width=0.5\linewidth]{1.png}
		\caption{Тестовый рисунок}
		\label{fig:mpr}
	\end{figure}
	
	Формула: $c^{2}=a^{2}+b^{2}$
	\begin{math}
		c^2 = a^2 + b_1^2
	\end{math}
		\pagebreak
	
	\section{123}

	\section*{ЗАКЛЮЧЕНИЕ}
		\addcontentsline{toc}{section}{ЗАКЛЮЧЕНИЕ}
		ТЕКСТ ВВЕДЕНИЯ
		\\ ТЕКСТ
		\pagebreak

	\section*{СПИСОК ИСПОЛЬЗОВАННЫХ ИСТОЧНИКОВ}
		\addcontentsline{toc}{section}{СПИСОК ИСПОЛЬЗОВАННЫХ ИСТОЧНИКОВ}

		\pagebreak
\end{document}
