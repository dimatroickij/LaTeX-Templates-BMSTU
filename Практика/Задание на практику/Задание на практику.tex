% !TEX program = xelatex
%----------------------- Преамбула -----------------------
\documentclass[utf8x, 10pt, oneside, a4paper]{article}

\usepackage{extsizes} % Для добавления в параметры класса документа 14pt
\usepackage{extsizes}
\usepackage{setspace}
\singlespacing
% Для работы с русским языком и шрифтом Times New Roman по-умолчанию
\usepackage[russian]{babel}
\usepackage{fontspec}
\setmainfont{Times New Roman}

% ГОСТовские настройки для полей
\usepackage[left=25mm, right=15mm, top=20mm, bottom=20mm]{geometry}

% Дополнительное окружение для подписей
\usepackage{array}
\newenvironment{signstabular}[1][1]{
	\renewcommand*{\arraystretch}{#1}
	\tabular
}{
	\endtabular
}
% Годные пакеты для обычных действий
\usepackage{ulem} % Нормальное нижнее подчеркивание
\baselineskip=10pt
% ---------------------- Документ ----------------------
\begin{document}
	\thispagestyle{empty}
	\begin{center}
		\textbf{Министерство науки и высшего образования Российской Федерации 
		\\ Федеральное государственное бюджетное образовательное учреждение 
		\\ высшего образования 
		\\ <<Московский государственный технический университет имени Н.Э. Баумана
		\\ (национальный исследовательский университет)>>
		\\ (МГТУ им. Н.Э. Баумана)}
		\fontsize{12pt}{0.15\baselineskip}\selectfont
		\noindent \makebox[\linewidth]{\rule{\textwidth}{4pt}} \makebox[\linewidth]{\rule{\textwidth}{1pt}}
	\end{center}	
	\normalsize
	\begin{flushright}
		УТВЕРЖДАЮ \hspace*{1.4cm}
		\\
		Заведующий кафедрой \uline{\hspace*{0.5cm}}\uline{ИУ6}\uline{\hspace*{0.5cm}} 
		\\ \hfill \textbf{<Должность>} \uline{\hspace*{2.5cm}} \ \uline{А.В. Пролетарский} 
		\\ \hfill
		\\ <<\uline{\hspace*{1cm}}>> \uline{\hspace*{2.5cm}} 20\ \ \   г. % момент подписания - начало семестра, формат даты: хх.хх.20хх
	\end{flushright}

	\begin{center}
		\fontsize{18pt}{\baselineskip}\selectfont \textbf{ЗАДАНИЕ}
		\\ \fontsize{16pt}{\baselineskip}\selectfont \textbf{на <учебную> / <производственную> практику} % либо одно, либо другое!!!
	\end{center}

	\normalsize

	\begin{flushleft}
		по теме  \uline{<Тема>\hfill}
		\baselineskip=17pt
		\\ \uline{ \hfill}
		\\ \uline{ \hfill}
		\\ \baselineskip=9pt\hfill
		\\ \baselineskip=10pt Студент группы  \uline{\hspace*{0.5cm}}\uline{Группа <Шифр>}\uline{\hspace*{0.5cm}}
		\center \baselineskip=7pt \uline{\hfill <ФИО> \hfill} 
		\\ \fontsize{10pt}{\baselineskip}\selectfont(фамилия, имя, отчество)
		\baselineskip=15pt
		 \\Тип практики: \uline{ \hfill} % По учебному плану: Практикум по программированию, Технологическая практика, Производственная практики, Преддипломная практика и т.п.
		\baselineskip=15pt
		\\  Название предприятия: \uline{ \hfill} % Для учебной практики: НУК ИУ МГТУ им. Н.Э. Баумана
		\\ \hfill
		\\ \textbf{\textit{Техническое задание}} \uline{\hfill}
		\\ \uline{\hfill}
		\\ \uline{\hfill}
		\\ \uline{\hfill}
		\\ \uline{\hfill}
		\\ \uline{\hfill}
		\\ \uline{\hfill}
	\end{flushleft}

	\begin{flushleft}
		\baselineskip=14pt
		\textbf{\textit{Оформление отчёта по практике:}}
		
		Расчетно-пояснительная записка на <25-30> листах формата А4. % Обязательно диапазон, например 20-25 или 25-30

		Перечень графического (иллюстративного) материала (чертежи, плакаты, слайды)
		% Не предусмотрен или  Необходимый иллюстративный графический материал включить в качестве рисунков в расчетно-пояснительную записку
		\\ \uline{\hfill} \baselineskip=15pt
		\\ \uline{\hfill}
		\\ \uline{\hfill}
	\end{flushleft}

	\begin{flushleft}
		Дата выдачи задания <<\uline{<День>}>> \uline{<Месяц>} \uline{<Год>} г. % Для распределенных практик обязательно начало семестра, для летних – дата первого дня практики
	\end{flushleft}

	\begin{table}[h!]
		\centering
		\begin{signstabular}[0.7]{p{8.6cm} >{\centering\arraybackslash}p{3.2cm} >{\centering\arraybackslash}p{3.8cm}}
			\textbf{Руководитель \ практики} & \uline{\hspace*{3.2cm}} & \uline{\hfill <И.О. Фамилия> \hfill} \\
			\rule{0pt}{0pt} & \fontsize{8pt}{\baselineskip}\selectfont (Подпись, дата) & \fontsize{8pt}{\baselineskip}\selectfont (И.О. Фамилия)
		%Дата обязательна, до утверждения заведующим кафедрой, формат даты: хх.хх.20хх

		\end{signstabular}

		\begin{signstabular}[0.7]{p{8.6cm} >{\centering\arraybackslash} >{\centering \arraybackslash}p{3.2cm} >{\centering\arraybackslash}p{3.8cm}}
			\textbf{Студент} & \uline{\hspace*{3.2cm}} & \uline{\hfill <И.О. Фамилия> \hfill} \\
			\rule{0pt}{0pt} & \fontsize{8pt}{\baselineskip}\selectfont (Подпись, дата) & \fontsize{8pt}{\baselineskip}\selectfont (И.О. Фамилия)
		\end{signstabular}
	\end{table}

	\begin{flushleft}
		\fontsize{11pt}{\baselineskip}\selectfont
		\uline{Примечание:} Задание оформляется в двух экземплярах -- один выдается студенту, второй хранится на кафедре
	\end{flushleft}
\end{document}
