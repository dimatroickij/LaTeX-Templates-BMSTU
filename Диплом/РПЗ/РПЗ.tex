% !TEX program = xelatex
%----------------------- Преамбула -----------------------
\documentclass[utf8x, 14pt, oneside, a4paper]{article}
\usepackage[left=30mm,right=15mm,top=20mm,bottom=20mm]{templateReportBMSTU}

% ---------------------- Документ ----------------------
\begin{document}
	\begin{titlepage}
		\noindent\begin{minipage}{0.05\textwidth}
			\includegraphics[scale=0.3]{1.png}
		\end{minipage}
		\hfill
		\begin{minipage}{0.85\textwidth}\raggedleft
			\begin{center}
				\fontsize{12pt}{0.3\baselineskip}\selectfont \textbf{Министерство науки и высшего образования Российской Федерации \\ Федеральное государственное бюджетное образовательное учреждение \\ высшего образования \\ <<Московский государственный технический университет \\ имени Н.Э. Баумана \\ (национальный исследовательский университет)>> \\ (МГТУ им. Н.Э. Баумана)}
			\end{center}
		\end{minipage}

		\begin{center}
			\fontsize{12pt}{0.1\baselineskip}\selectfont
			\noindent\makebox[\linewidth]{\rule{\textwidth}{4pt}} \makebox[\linewidth]{\rule{\textwidth}{1pt}}
		\end{center}

		\begin{flushleft}
			\fontsize{12pt}{0.8\baselineskip}\selectfont 
			
			ФАКУЛЬТЕТ \uline{<<\textbf{<Факультет (полностью)>}>> \hfill}

			КАФЕДРА \uline{\hspace{4mm} <<\textbf{<Кафедра (полностью)>}>> \hfill}
		\end{flushleft}

		\vfill

		\begin{center}
			\fontsize{20pt}{\baselineskip}\selectfont

			\textbf{РАСЧЕТНО-ПОЯСНИТЕЛЬНАЯ ЗАПИСКА}

			\textbf{\textit{К ВЫПУСКНОЙ КВАЛИФИКАЦИОННОЙ РАБОТЕ}}

			\textbf{\textit{НА ТЕМУ:}}
		\end{center}

		\begin{center}
			\fontsize{18pt}{0.6cm}\selectfont 
			
			\uline{\hfill}
	
			\uline{\hfill}
	
			\uline{\hfill}
	
			\uline{\hfill}
	
			\uline{\hfill}
		\end{center}

		\vfill

		\begin{table}[h!]
			\fontsize{12pt}{0.7\baselineskip}\selectfont
			\centering
			\begin{signstabular}[0.7]{p{7.25cm} >{\centering\arraybackslash}p{4cm} >{\centering\arraybackslash}p{4cm}}
				Студент группы \textbf{<Группа (Шифр)>} & \uline{\hspace*{4cm}} & \uline{\hfill \textbf{<И.О. Фамилия>} \hfill} \\
				& \scriptsize (Подпись, дата) & \scriptsize (И.О. Фамилия)
			\end{signstabular}

			\vspace{\baselineskip}

			\begin{signstabular}[0.7]{p{7.25cm} >{\centering\arraybackslash}p{4cm} >{\centering\arraybackslash}p{4cm}}
				Руководитель ВКР & \uline{\hspace*{4cm}} & \uline{\hfill \textbf{<И.О. Фамилия>} \hfill} \\
				& \scriptsize (Подпись, дата) & \scriptsize (И.О. Фамилия)
			\end{signstabular}

			\vspace{\baselineskip}

			\begin{signstabular}[0.7]{p{7.25cm} >{\centering\arraybackslash}p{4cm} >{\centering\arraybackslash}p{4cm}}
				Нормоконтроллер & \uline{\hspace*{4cm}} & \uline{\hfill \textbf{<И.О. Фамилия>} \hfill} \\
				& \scriptsize (Подпись, дата) & \scriptsize (И.О. Фамилия)
			\end{signstabular}
		\end{table}

		\vfill

		\begin{center}
			\normalsize \textit{\textbf{<Год>} г.}
		\end{center}
	\end{titlepage}

	\normalsize
	\setcounter{page}{4}
	\section*{АННОТАЦИЯ}

	\pagebreak

	\section*{РЕФЕРАТ}
		\begin{center}
			Расчетно-пояснительная записка \pageref{LastPage} с., \totalfigures\ рис., \totaltables\ табл., Х ист., Х прил.

			\textbf{<Ключевые слова>}
		\end{center}

		\pagebreak

	% Переопределяем название \toc и выводим сам \toc
	\renewcommand{\contentsname}{\normalsize\bfseries\centering СОДЕРЖАНИЕ}
	\small
	\tableofcontents
	\normalsize

		\pagebreak

	\section*{ОПРЕДЕЛЕНИЯ, ОБОЗНАЧЕНИЯ И СОКРАЩЕНИЯ}

		\pagebreak

	\section*{ВВЕДЕНИЕ}
		\addcontentsline{toc}{section}{ВВЕДЕНИЕ}

		\pagebreak

	\section{<Название исследовательской части>}
	ПРИМЕР НУМЕРОВАННОГО СПИСКА:
	\begin{enumerate} 
  		\item первый элемент первого уровня содержит список 
		\item  второй элемент первого уровня
	\end{enumerate}
	Пример ссылки на раздел: Подробная информация в разделе \ref{section1}
	\\ ПРИМЕР МАРКИРОВАННОГО СПИСКА:
	\begin{itemize}
		\item Each list item is ...
		\item Lists can be nested ...
			\begin{itemize}
				\item The maximum ...
				\item Switching ...
			\end{itemize}
		\item And so on.
	\end{itemize}
	
	Листинг \thenumberlisting \ - Код программы
	\begin{lstlisting}
		c = x^2
		print('test listing')
	\end{lstlisting}
		\pagebreak

	\section{<Название конструкторской части>}
	\begin{tabular}{ | p{100pt} | r | c | }
		\hline
		ИСЗ & Дата запуска & Масса, кг  \\ \hline
		Спутник-1 & 4 октября 1957 & 83.6 \\
		Спутник-2 & 3 ноября 1957 & 508.3 \\
		Эксплорер-1 & 1 февраля 1958 & 21.5 \\
		\hline
	\end{tabular}

	\begin{figure}[h]
		\centering
		\includegraphics[width=0.5\linewidth]{1.png}
		\caption{Тестовый рисунок}
		\label{fig:mpr}
	\end{figure}
	
	Формула: $c^{2}=a^{2}+b^{2}$
	\begin{math}
		c^2 = a^2 + b_1^2
	\end{math}
		\pagebreak

	\section{<Название технологической части>}

		\pagebreak

	\section*{ЗАКЛЮЧЕНИЕ}
		\addcontentsline{toc}{section}{ЗАКЛЮЧЕНИЕ}

		\pagebreak

	\section*{СПИСОК ИСПОЛЬЗОВАННЫХ ИСТОЧНИКОВ}
		\addcontentsline{toc}{section}{СПИСОК ИСПОЛЬЗОВАННЫХ ИСТОЧНИКОВ}

		\pagebreak
\end{document}
