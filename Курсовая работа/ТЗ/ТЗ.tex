% !TEX program = xelatex
%----------------------- Преамбула -----------------------
\documentclass[utf8x, 12pt, oneside, a4paper]{article}
\usepackage[left=30mm,right=15mm,top=20mm,bottom=20mm]{templateReportBMSTU}

% ---------------------- Документ ----------------------
\begin{document}
	\begin{titlepage}
		\begin{center}
			\fontsize{12pt}{0.4\baselineskip}\selectfont \textbf{Министерство науки и высшего образования Российской Федерации 
			\\ Федеральное государственное бюджетное образовательное учреждение 
			\\ высшего образования 
			\\ <<Московский государственный технический университет имени Н.Э. Баумана
			\\ (национальный исследовательский университет)>>
			\\ (МГТУ им. Н.Э. Баумана)}
		
			\fontsize{12pt}{0.2\baselineskip}\selectfont
			\noindent \makebox[\linewidth]{\rule{\textwidth}{4pt}} \makebox[\linewidth]{\rule{\textwidth}{1pt}}
		\end{center}	

		\begin{flushright}
			\fontsize{12pt}{0.5\baselineskip}\selectfont
			УТВЕРЖДАЮ \hspace*{1.4cm}
		
			\fontsize{12pt}{\baselineskip}\selectfont
			Заведующий кафедрой \uline{\hspace*{0.5cm}}\uline{ИУ6}\uline{\hspace*{0.5cm}} 
			\\ \hfill \textbf{<Должность>} \uline{\hspace*{2.5cm}} \ \uline{А.В. Пролетарский} 
			\\ <<\uline{\hspace*{1cm}}>> \uline{\hspace*{2.5cm}} 20\ \ \   г.
		\end{flushright}

		\vfill
		\begin{center}
			\fontsize{14pt}{\baselineskip}\selectfont <НАИМЕНОВАНИЕ ПРОГРАММНОГО ПРОДУКТА>
		\end{center}
		\center
		Техническое задание
		\center Листов <Количество>
		\vfill

		\begin{table}[h!]
			\fontsize{12pt}{0.7\baselineskip}\selectfont
			\centering
			\begin{signstabular}[0.7]{p{3.25cm} >  {\centering\arraybackslash}p{4cm} > {\centering\arraybackslash}p{4cm} > {\centering\arraybackslash}p{4cm}}
				Студент & \uline{\hspace*{4cm}} & \uline{\hspace*{4cm}} & \uline{\hfill \textbf{<И.О. Фамилия>} \hfill} 
				\\ & \scriptsize (Группа) & \scriptsize (Подпись, дата) & \scriptsize (И.О. Фамилия)
				% дата обязательна, формат даты хх.хх.20хх
			\end{signstabular}

			\vspace{\baselineskip}

			\begin{signstabular}[0.7]{p{3.25cm} >  {\centering\arraybackslash}p{4cm} > {\centering\arraybackslash}p{4cm} > {\centering\arraybackslash}p{4cm}}
				Руководитель & \ & \uline{\hspace*{4cm}} & \uline{\hfill \textbf{<И.О. Фамилия>} \hfill} 
				\\ & \ & \scriptsize (Подпись, дата) & \scriptsize (И.О. Фамилия)
				% дата обязательна, формат даты хх.хх.20хх
			\end{signstabular}
		\end{table}

	
		\begin{center}
			\normalsize \textit{\textbf{<Год>} г.}
		\end{center}	
	\end{titlepage}
	\normalsize
	\setcounter{page}{2}
	\pagebreak

	\section{ВВЕДЕНИЕ}
		Настоящее техническое задание распространяется на разработку \textbf{программы,  программного комплекса, программной системы> <наименование> [<шифр>]},  используемой для \textbf{<описание функционального назначения>} и предназначенной для \textbf{<описание области применения и возможных пользователей>}.

\textbf{<Далее введение должно продемонстрировать актуальность данной разработки и показать, какое место эта разработка занимает в ряду подобных.>}
	
\section{ОСНОВАНИЯ ДЛЯ РАЗРАБОТКИ}
	\textbf{<Наименование>} разрабатывается \textbf{< в соответствии с договором, приказом, распоряжением, заявкой и т.п., если разработки соответствует реальной работе студента, или по личной инициативе автора>}.

	\section{НАЗНАЧЕНИЕ РАЗРАБОТКИ}
Основное назначение \textbf{<наименование>} заключается в \textbf{<описание функционального и эксплуатационного назначения программного продукта с указанием категорий пользователей>}.

	\section{ИСХОДНЫЕ ДАННЫЕ, ЦЕЛИ И ЗАДАЧИ}
	\subsection{Исходные данные} \label{4}
	\subsubsection{Исходными данными для разработки являются следующие материалы:}
	4.1.1.1. \textbf{<перечень работ или письменных документов, содержащих исходные данные для разра-ботки> …}

	4.1.1.2. \textbf{<перечень прототипов> …}

	4.1.1.3. \textbf{…}
	\subsection{Цель работы}
	Целью работы является \textbf{(дистрибутивная версия, прототип, проект и т.п.)>} \textbf{<Наименование>} для \textbf{<кратко функциональное и эксплуатационное назначение>}.

	\subsection{Решаемые задачи}
	\subsubsection{Выбор \textbf{<модели жизненного цикла, архитектуры, подхода, технологии, методов, стандартов и средств разработки, если они не указаны в техническом задании>}.}

	\subsubsection{\textbf{Анализ требований технического задания с точки зрения выбранной технологии и уточнение требований к информационной системе: техническим средствам, внешним интерфейсам, а также к надежности и безопасности.}}
	\hfill
	\subsubsection{\textbf{Исследование предметной области  – разработка или выбор моделей, описывающих предметную область, или математическая постановка основных задач и/или выбор методов решения этих задач.}}

	\subsubsection{\textbf{Определение архитектуры информационной системы: разработка ее структуры; определение набора необходимого оборудования, программного обеспечения и процессов обслуживания.}}

	\subsubsection{Анализ требований технического задания и разработка спецификаций проектируемого программного обеспечения.}

	\subsubsection{Разработка структуры программного обеспечения и определение спецификаций его компонентов.}

	\subsubsection{Проектирование компонентов программного продукта \textbf{<отдельно указать, если есть, базы данных, подсистемы и т.п.>.}}

	\subsubsection{Реализация компонентов с использованием выбранных средств и их автономное тестирование.}

	\subsubsection{Сборка программного обеспечения и его комплексное тестирование.}

	\subsubsection{Оценочное тестирование программного обеспечения \textbf{<указать конкретно виды тестирования, например, тестирование удобства использования, тестирование на предельных нагрузках, тестирование на предельных нагрузках и т.п.>.}}

	\section{ТРЕБОВАНИЯ \textbf{К ПРОГРАММЕ ИЛИ ПРОГРАММНОМУ ИЗДЕЛИЮ}}

	\subsection{Требования к функциональным характеристикам}

	\subsubsection{Выполняемые функции:}
	5.1.1.1. Для пользователя:
	\textbf{
	\begin{itemize}
		\item функция 1;
		\item функция 2 и т. д.;
	\end{itemize}}

	\textbf{5.1.1.2. Для администратора системы (если он предусматривается):
	\begin{itemize}
		\item функция 1;
		\item функция 2 и т. д.;
	\end{itemize}}

	\subsubsection{\textbf{Исходные данные:}}

	\begin{itemize}
		\item \textbf{информация 1;}
		\item \textbf{информация 2 и т. д.;}
	\end{itemize}
	\subsubsection{\textbf{Результаты:}}
	\begin{itemize}
		\item \textbf{информация 1;}
		\item \textbf{информация 2 и т. д.;}
	\end{itemize}
	\textbf{<здесь же указывают максимально допустимое время ответа системы, максимальный объем используемой оперативной и/или внешней памяти и т.п.>}

	\subsection{Требования к надежности}

	\subsubsection{Предусмотреть контроль вводимой информации.}
	
	\subsubsection{Предусмотреть защиту от некорректных действий пользователя.}

	\subsubsection{\textbf{Обеспечить целостность информации в базе данных.}}
	\textbf{<Кроме того, можно указать требования к восстановлению после сбоев, например, время восстановления системы, наличие контрольных точек, резервных копий полученных промежуточных результатов и т.п.> }

	\subsection{Условия эксплуатации}

	\subsubsection{Условия эксплуатации в соответствие с СанПиН 2.2.2/2.4.1340-03.}
	
	\subsubsection{\textbf{Обслуживание}}

	\subsubsection{\textbf{Обслуживающий персонал}}

	\textbf{<при необходимости указывают основные операции обслуживания, необходимые количество и квалификацию персонала>}

	\subsection{Требования к составу и параметрам технических средств}


	\subsubsection{Программное обеспечение должно функционировать на IBM-совместимых персональных компьютерах.}

	\subsubsection{Минимальная конфигурация технических средств:}
	5.4.2.1. Тип процессора .............. \textbf{<Pentium>}
	
	5.4.2.2. Объем ОЗУ ............. \textbf{<XXX Мб>}

	\textbf{5.4.2.3. и т. п.}

	\subsection{Требования к информационной и программной совместимости}

	\subsubsection{Программное обеспечение должно работать под управлением операционных систем семейства \textbf{WIN32 (64) (Windows 10 и т.д.)}.}

	\subsubsection{\textbf{Входные данные должны быть представлены в следующем формате: <описание формата> (только для подсистем).}}

	\subsubsection{\textbf{Результаты должны быть представлены в следующем формате: <описание формата> (только для подсистем).}}
	
	\subsubsection{\textbf{Программное обеспечение должно <описание интерфейса (протокола) с другим программным обеспечением>.}}
	\textbf{<Можно, но лучше не надо, также указать средства: язык и среду разработки, а также требования к защите информации>}

	\subsection{Требования к маркировке и упаковке}
	Требования к маркировке и упаковке не предъявляются.
	
	\subsection{Требования к транспортированию и хранению}
	Требования к транспортировке и хранению не предъявляются.
	
	\subsection{Специальные требования}
	Сгенерировать установочную версию программного обеспечения.

		
	\section{ТРЕБОВАНИЯ К ПРОГРАММНОЙ ДОКУМЕНТАЦИИ}

	\subsection{Разрабатываемые программные модули должны быть самодокументированы, т.е. тексты программ должны содержать все необходимые комментарии.}
	
	\subsection{Разрабатываемое программное обеспечение должно включать справочную систему.}
	
	\subsection{В состав сопровождающей документации должны входить:}

	\subsubsection{Расчетно-пояснительная записка на \textbf{50-60} листах формата А4 (без приложений 6.3.2, 6.3.3 и 6.3.4)} % для бакалавров – 50–60, для магистров – 95-105

	\subsubsection{Техническое задание (Приложение A).}

	\subsubsection{\textbf{Руководство пользователя (Приложение Б) – при необходимости.}}

	\subsubsection{\textbf{Руководство системного программиста (Приложение В) – при необходимости.}}

	\subsection{Графическая часть должна быть выполнена на \textbf{6} листах формата А1 (копии формата А3, А4 включить в качестве приложений к расчетно-пояснительной записке):} % Для бакалавров – 6, для магистров - 10

	\subsubsection{\textbf{Схема структурная информационной системы.}}

	\subsubsection{\textbf{Спецификация функциональная.}}

	\subsubsection{Схема структурная программного обеспечения.}

	\subsubsection{\textbf{Схема функциональная программного обеспечения.}}

	\subsubsection{\textbf{Функциональная диаграмма программного обеспечения (или его части).}}

	\subsubsection{\textbf{Диаграмма потоков данных программного обеспечения или его части.}}

	\subsubsection{\textbf{Диаграммы (схемы) компонентов структур данных.}}

	\subsubsection{\textbf{Структуры (модели) знаний.}}

	\subsubsection{\textbf{Схемы (модели) процессов (методов формирования результатов, механизмы выводов и т.п.).}}

	\subsubsection{\textbf{Схемы (модели) синтаксического, семантического представления (языка входных и вы-ходных сообщений и т.д.).}}

	\subsubsection{\textbf{Диаграмма вариантов использования.}}

	\subsubsection{\textbf{Концептуальная модель предметной области.}}

	\subsubsection{\textbf{Схемы взаимодействия объектов, объектная декомпозиция.}}
	
	\subsubsection{\textbf{Схемы структурные компонент, например, даталогическая и/или инфологическая схемы базы данных.}}

	\subsubsection{\textbf{Схема взаимодействия модулей.}}

	\subsubsection{\textbf{Диаграммы классов предметной области и/или интерфейсной части программного обес-печения.}}

	\subsubsection{\textbf{Граф (диаграмма) состояний интерфейса.}}

	\subsubsection{\textbf{Структурная схема меню.}}

	\subsubsection{\textbf{Графы диалогов.}}
	\hfill
	\subsubsection{\textbf{Формы интерфейса.}}

	\subsubsection{\textbf{Схемы алгоритмов модулей (подпрограмм).}}
	
	\subsubsection{\textbf{Диаграммы компоновки программных компонентов.}}

	\subsubsection{\textbf{Диаграммы размещения программных компонентов.}}

	\subsubsection{\textbf{Таблица характеристик (инструментальных средств разработки, языка, среды программирования, средств автоматизации разработки, методов тестирования, подхода к разработке).}}

	\subsubsection{\textbf{Таблицы тестов.}}

	\subsubsection{\textbf{Схемы алгоритмов тестовых программ.}}

	\subsubsection{\textbf{Схема алгоритма тестирования.}}

	\subsubsection{\textbf{Схема процесса разработки программного продукта (при различных технологиях, напри-мер, при структурном, объектном, нисходящем, восходящем подходах и т.п.).}}

	\subsubsection{\textbf{Таблица характеристик качества программного обеспечения.}}

	\section{ТЕХНИКО-ЭКОНОМИЧЕСКИЕ ПОКАЗАТЕЛИ}
	Выполнить технико-экономическое обоснование разработки.

	\section{СТАДИИ И ЭТАПЫ РАЗРАБОТКИ}
\begin{flushleft}
\begin{tabular}{|p{0.83cm} |>  {\centering\arraybackslash}p{7.62cm} |> {\centering\arraybackslash}p{3.75cm} |> {\centering\arraybackslash}p{4.19cm}|}
 			\hline
 			№ & Название этапа & Срок, даты, \% & Отчётность \\ 
			\hline
 			1. & Разработка технического задания & 2.02.2019 -20.02.2020 5 \%  & Утвержденное техническое задание \\ 
			\hline
 			2. & \textbf{Анализ требований и уточнение спецификаций (эскизный проект)} &  \textbf{...} & \textbf{Спецификации программного обеспечения. } \\ 
 			\hline
			3. & \textbf{Проектирование структуры программного обеспечения, проектирование  компонентов (технический проект)} & \textbf{...} & \textbf{Схема структурная системы и спецификации компонентов. Проектная документация: схемы, диаграммы и т.п.} \\
			\hline
			4. &  \textbf{Реализация компонентов и автономное тестирование компонентов. Сборка и комплексное тестирование. Оценочное тестирование и (рабочий проект).} & \textbf{...} & \textbf{Тексты программных компонентов. Тесты, результаты тестирования.} \\
			\hline
			5. & Разработка документации.  & \textbf{...} - 25.05.2020 8 \% & Расчетно-пояснительная записка. \\
			\hline
			6. & Прохождение нормоконтроля, проверка на антиплагиат, получение рецензии, подготовка доклада  и предзащита.  & 25.05.2020-6.06.2020 5 \% & Иллюстративный материал, доклад, рецензия, справки о нормоконтроле и проценте плагиата. \\
			\hline
			7. & Защита выпускной квалификационной работы. & 8.06.2020-04.07.2020 2 \% & \hfill \\
			\hline
		\end{tabular}
\end{flushleft}
	\section{ПОРЯДОК КОНТРОЛЯ И ПРИЕМКИ}
	
	\subsection{Порядок контроля}
	Контроль выполнения осуществляется руководителем еженедельно.

	\subsection{Порядок защиты}
	Защита осуществляется перед государственной экзаменационной комиссией (ГЭК).
	
	\subsection{Срок защиты}
	Срок защиты определяется в соответствии с планом заседаний ГЭК.

	\section{ПРИМЕЧАНИЕ}
	В процессе выполнения работы возможно уточнение отдельных требований технического задания по взаимному согласованию руководителя и исполнителя.
\end{document}
	


