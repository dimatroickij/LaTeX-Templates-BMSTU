% !TEX program = xelatex
%----------------------- Преамбула -----------------------
\documentclass[11pt]{templateTaskBMSTU}

% ---------------------- Документ ----------------------
\begin{document}
	\thispagestyle{empty}
	\begin{center}
		\fontsize{11pt}{0.3\baselineskip}\selectfont \textbf{Министерство науки и высшего образования Российской Федерации 
		\\ Федеральное государственное бюджетное образовательное учреждение 
		\\ высшего образования 
		\\ <<Московский государственный технический университет имени Н.Э. Баумана
		\\ (национальный исследовательский университет)>>
		\\ (МГТУ им. Н.Э. Баумана)}
		
		\fontsize{12pt}{0.5\baselineskip}\selectfont
		\noindent \makebox[\linewidth]{\rule{\textwidth}{4pt}} \makebox[\linewidth]{\rule{\textwidth}{1pt}}
	\end{center}	
	
	\begin{flushright}
		\fontsize{12pt}{\baselineskip}\selectfont
		УТВЕРЖДАЮ \hspace*{1.4cm}

		\fontsize{12pt}{\baselineskip}\selectfont
		Заведующий кафедрой \uline{\hspace*{0.5cm}}\uline{ИУ6}\uline{\hspace*{0.5cm}} 
		\\ \
		\\ \hfill \textbf{<Должность>} \uline{\hspace*{2.5cm}} \ \uline{А.В. Пролетарский} 
		\\ <<\uline{\hspace*{1cm}}>> \uline{\hspace*{2.5cm}} 20\ \ \   г. % момент подписания - начало семестра
	\end{flushright}
	\
	\begin{center}
		\fontsize{18pt}{\baselineskip}\selectfont \textbf{ЗАДАНИЕ}
		\\ \fontsize{16pt}{\baselineskip}\selectfont \textbf{на выполнение курсовой работы}
	\end{center}

	\normalsize

	\begin{flushleft}
		по дисциплине  \uline{\hspace*{0.5cm}}\uline{<Дисциплина>}\uline{\hspace*{0.5cm}}
		
		\

		Студент группы  \uline{\hspace*{0.5cm}}\uline{Группа <Шифр>}\uline{\hspace*{0.5cm}}
		\center \uline{\hfill <ФИО> \hfill} 
		\\ \fontsize{10pt}{\baselineskip}\selectfont(фамилия, имя, отчество)
	\end{flushleft}

	\begin{flushleft}
		Тема курсовой работы: \uline{ \hfill}
		\\ \uline{ \hfill}
		\\ \uline{ \hfill}
	\end{flushleft}

	\begin{flushleft}
		Направленность КР (учебная, исследовательская, практическая, производственная, др.)
		\\ \uline{\hfill}
		\\ Источник тематики (кафедра, предприятие, НИР): \uline{\hfill}
	\end{flushleft}

	\begin{flushleft}
		График выполнения работы:  25\% к 4 нед., 50\% к 7 нед., 75\% к 11 нед., 100\% к 14 нед.
	\end{flushleft}

	\begin{flushleft}
		\textbf{\textit{Задание}} \uline{\hfill} % текст задания или смотри техническое задание в приложении А
		
		\uline{\hfill}
		
		\uline{\hfill}

		\uline{\hfill}

		\uline{\hfill}
	\end{flushleft}

	\begin{flushleft}
		\textbf{\textit{Оформление курсовой работы:}}
		
		Расчетно-пояснительная записка на 25-30 листах формата А4.

		Перечень графического (иллюстративного) материала (чертежи, плакаты, слайды)
		
		\uline{\hfill}

		\uline{\hfill}

		\uline{\hfill}

		\uline{\hfill}
	\end{flushleft}

	\begin{flushleft}
		Дата выдачи задания <<\uline{<День>}>> \uline{<Месяц>} \uline{<Год>} г. % начало семестра!
	\end{flushleft}

	\begin{table}[h!]
		\centering
		\begin{signstabular}[0.7]{p{8.6cm} >{\centering\arraybackslash}p{3.2cm} >{\centering\arraybackslash}p{3.8cm}}
			\textbf{Руководитель \ курсовой работы} & \uline{\hspace*{3.2cm}} & \uline{\hfill <И.О. Фамилия> \hfill} \\
			\rule{0pt}{0pt} & \fontsize{8pt}{\baselineskip}\selectfont (Подпись, дата) & \fontsize{8pt}{\baselineskip}\selectfont (И.О. Фамилия)
			% Дата обязательна, до утверждения заведующим кафедрой, формат даты: хх.хх.20хх
		\end{signstabular}

		\begin{signstabular}[0.7]{p{8.6cm} >{\centering\arraybackslash} >{\centering \arraybackslash}p{3.2cm} >{\centering\arraybackslash}p{3.8cm}}
			\textbf{Студент} & \uline{\hspace*{3.2cm}} & \uline{\hfill <И.О. Фамилия> \hfill} \\
			\rule{0pt}{0pt} & \fontsize{8pt}{\baselineskip}\selectfont (Подпись, дата) & \fontsize{8pt}{\baselineskip}\selectfont (И.О. Фамилия)
			% Дата обязательна, до утверждения заведующим кафедрой, формат даты: хх.хх.20хх
		\end{signstabular}
	\end{table}

	\begin{flushleft}
		\fontsize{12pt}{\baselineskip}\selectfont
		\uline{Примечание:} Задание оформляется в двух экземплярах -- один выдается студенту, второй хранится на кафедре
	\end{flushleft}
\end{document}
